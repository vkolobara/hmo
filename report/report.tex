%%%%%%%%%%%%%%%%%%%%%%%%%%%%%%%%%%%%%%%%%%%%%%%%%%%%%%%%%%%%%%%%%%%%%%
% LaTeX Example: Project Report
%
% Source: http://www.howtotex.com
%
% Feel free to distribute this example, but please keep the referral
% to howtotex.com
% Date: March 2011 
% 
%%%%%%%%%%%%%%%%%%%%%%%%%%%%%%%%%%%%%%%%%%%%%%%%%%%%%%%%%%%%%%%%%%%%%%
% How to use writeLaTeX: 
%
% You edit the source code here on the left, and the preview on the
% right shows you the result within a few seconds.
%
% Bookmark this page and share the URL with your co-authors. They can
% edit at the same time!
%
% You can upload figures, bibliographies, custom classes and
% styles using the files menu.
%
% If you're new to LaTeX, the wikibook is a great place to start:
% http://en.wikibooks.org/wiki/LaTeX
%
%%%%%%%%%%%%%%%%%%%%%%%%%%%%%%%%%%%%%%%%%%%%%%%%%%%%%%%%%%%%%%%%%%%%%%
% Edit the title below to update the display in My Documents
%\title{Project Report}
%
%%% Preamble
\documentclass[paper=a4, fontsize=11pt]{scrartcl}
\usepackage[utf8]{inputenc}
\usepackage[T1]{fontenc}
\usepackage{fourier}
\usepackage{caption}
\usepackage{algorithm}
\usepackage{algpseudocode}

\usepackage[english]{babel}															% English language/hyphenation
\usepackage[protrusion=true,expansion=true]{microtype}	
\usepackage{amsmath,amsfonts,amsthm} % Math packages
\usepackage[pdftex]{graphicx}	
\usepackage{url}


%%% Custom sectioning
\usepackage{sectsty}
\allsectionsfont{\centering \normalfont\scshape}


%%% Custom headers/footers (fancyhdr package)
\usepackage{fancyhdr}
\pagestyle{fancyplain}
\fancyhead{}											% No page header
\fancyfoot[L]{}											% Empty 
\fancyfoot[C]{}											% Empty
\fancyfoot[R]{\thepage}									% Pagenumbering
\renewcommand{\headrulewidth}{0pt}			% Remove header underlines
\renewcommand{\footrulewidth}{0pt}				% Remove footer underlines
\setlength{\headheight}{13.6pt}


%%% Equation and float numbering
\numberwithin{equation}{section}		% Equationnumbering: section.eq#
\numberwithin{figure}{section}			% Figurenumbering: section.fig#
\numberwithin{table}{section}				% Tablenumbering: section.tab#


%%% Maketitle metadata
\newcommand{\horrule}[1]{\rule{\linewidth}{#1}} 	% Horizontal rule

\title{
		%\vspace{-1in} 	
		\normalfont \normalsize \textsc{Fakultet elektrotehnike i računarstva} \\ [25pt]
		\horrule{0.5pt} \\[0.4cm]
		\huge Heurističke metode optimizacija \\
		\Large Usmjeravanje školskih autobusa \\
		\horrule{2pt} \\[0.5cm]
}
\author{
		\normalfont 								\normalsize
        Vinko Kolobara \\[-3pt]		\normalsize
        \today
}
\date{}


%%% Begin document
\begin{document}
\maketitle
\section{Opis problema}
\paragraph{}
Problem usmjeravanja školskih autobusa je varijacija problema usmjeravanja vozila u kojemu nemamo predefinirane postaje, nego moramo odrediti i postaje koje ćemo posjetiti, te formirati autobusne linije što ovisi o mjesto stanovanja pojedinog učenika (na koju stanicu ide koji učenik) i kapaciteta autobusa.
\paragraph{}

\textbf{Cilj} rješavanja problema je:
\begin{enumerate}
\item odrediti skup stanica koje će se posjetiti,
\item odrediti za svakog učenika do koje stanice treba hodati,
\item odrediti autobusne linije, tako da se minimizira ukupna udaljenost koju autobusi prelaze.
\end{enumerate}

Pritom vrijede neka \textbf{ograničenja i pretpostavke}:
\begin{enumerate}
\item svakom učeniku mora biti dodijeljena stanica čija je udaljenost manja od maksimalne dozvoljene udaljenosti,
\item svi autobusi imaju isti kapacitet koji ne smije biti premašen,
\item svaku postaju posjećuje samo jedan autobus, što znači da svi učenici koji idu na istu stanicu ne mogu ići različitim autobusima.
\end{enumerate}

Početna i završna točka svih autobusnih linija mora biti škola.

%% Možda dodaj i matematičku formulaciju tu


\section{Opis algoritma}
Za rješavanje navedenog problema, korišten je hibridni genetski algoritam.
\subsection{Prikaz rješenja}
Svaki kromosom sastoji se od:

\begin{itemize}
\item zapisa autobusnih linija, permutacijski vektor cijelih brojeva koji predstavlja redoslijed obilaska stanica. Dekodira se u više linija tako da, kada se vidi da je premašen kapacitet autobusa, počinje nova linija, a sve stanice koje nisu posjećene se jednostavno preskaču.
\item zapis dodjele stanica studentima, vektor cijelih brojeva koji označava koji student ide na koju stanicu.
\item broj studenata po stanicama, pomoćni vektor cijelih brojeva koji olakšava dodjeljivanje stanica.
\end{itemize}

\begin {table}[h]
\caption{Primjer kromosoma, kapacitet autobusa 3}
\begin{center}
\begin{tabular}{ | c | c | c | c | c |}
	\hline
	1	&	3	&	2	&	0	&	5 \\
	\hline
\end{tabular}

\caption*{Zapis autobusnih linija, dekodirat će se u dvije linije: 1, 3 -> 0}

\begin{tabular}{ | c | c | c | c | c |}
	\hline
	0	&	1	&	1	&	1	&	2 \\
	\hline
\end{tabular}

\caption*{Dodjela stanica studentima, student 1 ide na stanicu 0, student 2 na stanicu 1 \dots}

\begin{tabular}{ | c | c | c | c | c |}
	\hline
	1	&	3	&	1	&	0	&	0 \\
	\hline
\end{tabular}
\caption*{Broj studenata po stanicama, stanica 0 ima 1 studenta, stanica 1 3 studenta i stanica 2 1 studenta }

\end{center}
\end{table}

\subsection{Funkcija cilja}
Funkcija cilja je predstavljena zbrojem duljina svih linija (euklidska udaljenost). Kako nam je želja što manja ukupna udaljenost, ovo je problem minimizacije.

\subsection{Generiranje početnog rješenja}
Stvaranje početnog rješenja odvija se u dvije faze. U prvoj fazi, potrebno je dodijeliti stanice studentima, a druga faza odnosi se na određivanje autobusnih linija tako odabranom podskupu stanica.

\subsubsection{Dodjeljivanje stanica studentima}
Za ovaj dio koristi se GRASP i to tako da se svakom studentu nasumično dodijeli jedna od $\alpha$ stanica najbližih školi (nije nam bitno koliko jadni studenti pješače).

\subsubsection{Određivanje linija}
I za ovaj dio se koristi GRASP, i to tako što se prvo postave linije tako da je jedna stanica jedna linija. Nakon toga spajamo linije uz heuristiku:
\[h(i,j) = distanceToSchool(i) + distanceToSchool(j) - euclideanDistance(i,j)\]
Nakon toga, temeljem heuristike spajamo nasumično jedan od $\alpha$ najboljih parova.

\subsection{Kriterij zaustavljanja}
Kriterij zaustavljanja algoritma je 5 minuta bez unaprjeđivanja rješenja, ili 60 minuta ukupnog trajanja izvođenja.

\subsection{Tijek algoritma}
Algoritam je blaga varijacija na klasični genetski algoritam što se može vidjeti iz pseudokoda.

\begin{algorithm}[h]
\caption{Hibridni genetski algoritam}
\begin{algorithmic}
\Procedure{HGA}{}
\State stvori početnu populaciju pomoću GRASP-a
\State evaluiraj početnu populaciju
\While{!kriterijZaustavljanja}
  \If{iter mod 10000 }
    \State ubaci novo GRASP rješenje u populaciju
  \Else
    \State turnirska selekcija 
    \State križaj i mutiraj novu jedinku
    \State evaluiraj novu jedinku
    \State zamijeni najgoru jedinku turnira s novom jedinkom
  \EndIf
\EndWhile
\If{nema napretka}
  \State stvori novu populaciju
  \State dodaj najbolje rješenje iz stare u novu populaciju
\EndIf
\EndProcedure
\end{algorithmic}
\end{algorithm}

%%% End document
\end{document}
